%!TEX root = ./main.tex
\chapter{Prefácio}
\label{prefacio}

O objetivo deste relatório é apresentar a evolução e as mudanças do meu projeto de pesquisa no período de Outubro de 2021 à Outubro de 2022. Para auxiliar no processo geral de escrita da tese, lembro que alguns textos do relatório anterior e do exame de qualificação foram atualizados e serão reapresentados. 

% O capítulo de introdução sofreu pequenas modificações e as mudanças mais significantes estão na seção \ref{areaDeEstudo}, que apresenta alguns perfis tomográficos da região de interesse deste projeto e que são relevantes para o início das simulações sintéticas.

% O capítulo \ref{metodos} foi modificado para conter informações relevantes para as simulações que são descritas no capítulo \ref{simulacoes}, mas a teoria completa que escrevi está na documentação do \textit{Mandyoc}, disponível \textit{online} e no capítulo \ref{documentacao-mandyoc}.

% Em colaboração com a Dra. Agustina Pesce (\textit{Instituto Geofísico Sismológico Ing. Volponi, Universidad Nacional de San Juan, Argentina}), venho desenvolvendo cenários iniciais para simular a subducção da placa de Nazca em duas dimensões. Os códigos que estamos desenvolvendo estão no repositório do \textit{Github} (\url{https://github.com/aguspesce/subduction_model}). Mais informações sobre as simulações estão disponíveis no capítulo \ref{simulacoes} deste relatório. 

% O grande foco de trabalho deste período foi a publicação do código \textit{Mandyoc} e a finalização de sua documentação essencial. Nos últimos meses colaborei com Victor Sacek, Agustina Pesce e Rafael Monteiro da Silva para a publicação de um artigo que disponibiliza oficialmente o código \textit{Mandyoc}. Mais informações sobre a publicação do artigo e sua documentação estão no capítulo \ref{documentacao-mandyoc}.

Durante o último ano o foco deste projeto tem sido a publicação do código \textit{Mandyoc}, a finalização de sua documentação essencial, o exame de qualificação do doutorado e a realização de uma série de simulações de subdução em duas dimensões. 

Em colaboração com Victor Sacek, Agustina Pesce e Rafael Monteiro da Silva, o código \textit{Mandyoc} foi publicado na revista JOSS (\textit{Journal of Open Source Software}) com o título \textit{Mandyoc: A finite element code to simulate thermochemical convection in parallel}. A documentação do código está hospedada no \textit{link} \url{https://ggciag.github.io/mandyoc/}, mas seu desenvolvimento deve continuar tanto para facilitar a utilização do código quanto para adicionar e/ou atualizar funcionalidades.

Com a aprovação no exame de qualificação de doutorado na primeira parte de 2022, este relatório também procura apresentar algumas correções propostas para a monografia que foi entregue e cujo título é "Geodinâmica de subdução de placas oceânicas". Para evitar redundâncias, o texto foi adaptado e fundido aos capítulos base anteriores de introdução e métodos.

No capítulo \ref{simulacoes}, algumas das simulações 2D realizadas são apresentadas e discutidas. As simulações foram motivadas em grande parte pelo trabalho de \cite{strak2021thermo}, que fez um estudo numérico multiparamétrico da subdução uma placa oceânica sob uma placa continental. Seguindo a proposta, a subdução simulada ocorre de maneira livre, ou seja, a placa oceânica subduz devido ao seu peso e sob nenhuma força externa.

No capítulo \ref{artigo} são descritas as tarefas e objetivos atuais, além de suas justificativas, dando destaque aqui já à modificação do código \textit{Mandyoc}, que deve passar a considerar mudança de fase. Um novo artigo intitulado \textit{Computational geodynamics of South American plate: review and perspectives} também está em fase de revisão e é apresentado. O artigo é uma colaboração com Victor Sacek, Naomi Ussami, Agustina Pesce, Claudio Alejandro Salazar-Mora, Edgar Bueno dos Santos, Felipe Baiadori, João Pedro Macedo Silva, Rafael Monteiro da Silva e Tacio Cordeiro Bicudo.

Por fim, as mudanças gerais feitas ao cronograma do projeto são apresentadas no capítulo \ref{cronograma}.

