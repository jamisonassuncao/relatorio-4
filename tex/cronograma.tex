%!TEX root = ./main.tex
\chapter{Cronograma}
\label{cronograma}
\renewcommand{\thefootnote}{\arabic{footnote}}

Os objetivos destacados no relatório anterior foram modificados e o plano de pesquisa segue como mostra a tabela \ref{table:fixed-params}.

\begin{center}
    \setcaptionmargin{1cm}
    \scriptsize
    \begin{longtable}{lcccccccc}
        \caption[Plano de pesquisa]{Plano de pesquisa.}\\
        \hline \hline \\[-2ex]
        \multicolumn{1}{c}{Tarefa} &
        \multicolumn{1}{c}{2020a} &
        \multicolumn{1}{c}{2020b} &
        \multicolumn{1}{c}{2021a} &
        \multicolumn{1}{c}{2021b} &
        \multicolumn{1}{c}{2022a} &
        \multicolumn{1}{c}{2022b} &
        \multicolumn{1}{c}{2023a} &
        \multicolumn{1}{c}{2023b} 
        
        \\[0.5ex] \hline
        \\[-1.8ex]
        
        \endfirsthead
        
        \multicolumn{4}{c}{\footnotesize{{\slshape{{\tablename} \thetable{}}} - Continuação}}\\[0.5ex]
        
        \hline \hline\\[-2ex]
        
        \multicolumn{1}{c}{Tarefa} &
        \multicolumn{1}{c}{2020a} &
        \multicolumn{1}{c}{2020b} &
        \multicolumn{1}{c}{2021a} &
        \multicolumn{1}{c}{2021b} &
        \multicolumn{1}{c}{2022a} &
        \multicolumn{1}{c}{2022b} &
        \multicolumn{1}{c}{2023a} &
        \multicolumn{1}{c}{2023b} 
        
        \\[0.5ex] \hline
        \\[-1.8ex]
        
        \endhead
        
        \multicolumn{5}{l}{{\footnotesize{Continua na próxima página\ldots}}}\\
        \endfoot
        \hline
        
        \endlastfoot

		Disciplinas & 
            $\times$ & & & & & & & \\ 
		Familiarização numérica & 
			$\times$ & $\times$ & $\times$ & & & & & \\
		% Revisão Bibliográfica & 
			% $\times$ & $\times$ & $\times$ & $\times$ & & \\
		\textit{Benchmarks} &
            & $\times$ & $\times$ & & & & & \\
        Redação de artigo (\textit{Mandyoc}) &
            & & $\times$ & $\times$ & & & & \\
        Simulações sintéticas\footnotemark[1] &
			& & & $\times$ & $\times$ & & & \\
        Exame de Qualificação\footnotemark[2] & 
            & & & $\times$ & $\times$ & & & \\
		Redação de artigo (modelo sintético) & 
			& & & & $\times$ & $\times$ & & \\
		Simulações reais\footnotemark[3] & 
            & & & &  $\times$ & $\times$ & $\times$ & $\times$ \\
		Redação (tese e artigo) & 
			& & & & $\times$ &  $\times$ & $\times$ & $\times$ \\
		Apresentação Final & 
			& & & & & & & $\times$ \\
        
        \label{table:fixed-params}
    \end{longtable}
\end{center}

\footnotetext[1]{Geometria e estrutura térmica simples serão utilizadas.}
\footnotetext[2]{Inscrição no exame de qualificação em 2021b, e realização em 2022a.}
\footnotetext[3]{Geometria da placa e espessuras litosféricas de modelos globais e regionais serão utilizadas com a devida estrutura térmica.}

Com a publicação do artigo do \textit{Mandyoc} à \textit{JOSS}, o foco agora é a implementação da mudança de fase ao código \textit{Mandyoc}.

% A partir dos resultados obtidos, vou buscar correlacionar padrões e entender os vínculos entre as simulações sintéticas e as feições geológicas e geofísicas observadas.

% Nas próximas semanas, pretendo submeter as modificações que fiz no código \textit{Mandyoc} e também iniciar os testes 3D do caso sintético. Concomitantemente, devo incorporar em conjunto com o Prof. Dr. Victor Sacek as mudanças necessárias no código para considerar metamorfismo na placa em subducção.
